\documentclass{beamer}
\usepackage[UTF8,noindent]{ctexcap}
\usepackage{hyperref}
\usetheme{Berlin}
\usecolortheme{sustech}
\usepackage{enumitem,amssymb}
\newlist{todolist}{itemize}{2}
\setlist[todolist]{label=$\square$}
\usepackage{pifont}
\newcommand{\cmark}{\ding{51}}
\newcommand{\xmark}{\ding{55}}
\newcommand{\done}{\rlap{$\square$}{\raisebox{2pt}{\large\hspace{1pt}\cmark}}\hspace{-1.5pt}}
\newcommand{\wontfix}{\rlap{$\square$}{\large\hspace{1.5pt}\xmark}}

\newcommand{\tri}{$\blacktriangleright$}

\begin{document}

\begin{frame}
	\title{rCore 到 zCore 的功能迁移组\ 结题报告}
	\author{郑权\ 郑昱笙\ 许善朴\ 李宇}
	\date{\today}
	\titlepage
\end{frame}

\begin{frame}
	\frametitle{目录}
	\tableofcontents
\end{frame}

\section{开始之前}
\begin{frame}
	\frametitle{zCore 状况}
	\begin{itemize}
		\item [\tri] zCore 中 Linux 相关模块的文档和单元测试相对缺失
		\item [\tri] 文件系统除以下部分基本完成
		\begin{itemize}
			\item [\tri] stdin 尚未实现
			\item [\tri] IO 多路复用尚未实现
			\item [\tri] 时间相关模块尚未实现
		\end{itemize}
		\item [\tri] 进程间通信机制暂时缺失
		\item [\tri] 信号机制暂时缺失
		\item [\tri] Linux 测试用例暂时缺失
		\item [\tri] 仅能运行简单的 busybox 等小程序
	\end{itemize}
\end{frame}

\begin{frame}
	\frametitle{组内分工}
	郑昱笙
	\begin{itemize}
		\item [\tri] 尝试完善 Linux 系统调用的单元测试及 libc-test
		\item [\tri] 以测试驱动开发,尝试尽可能多地移植 rCore 相关功能
	\end{itemize}
	李宇
	\begin{itemize}
		\item [\tri] 尝试移植 rCore 支持的程序,如 GCC,Rust 工具链等
		\item [\tri] 以移植驱动开发,尝试移植及修复 zCore 相关功能
	\end{itemize}
\end{frame}

\section{主要成果}
\begin{frame}
	\frametitle{阶段进展}
	\begin{itemize}
		\begin{todolist}
			\item [\done] 完善代码文档
			\item [\done] 完善 Linux 文件系统相关功能
			\item [\done] 完善 Linux 进程通信相关功能
			\item [\done] 添加系统调用的单元测试
			\item [\done] 实现 stdin
			\item [\done] 移植 shell
			\item [\done] 移植 GCC
			\item 移植 Rust 工具链
		\end{todolist}
	\end{itemize}
\end{frame}

\begin{frame}
	\frametitle{成果概述}
	Linux 相关模块测试覆盖率变化
	\begin{itemize}
		\item [\tri] linux-loader\ 74.63 $\rightarrow$ 87.88
		\item [\tri] linux-syscall\ 18.68 $\rightarrow$ 61.55
		\item [\tri] linux-object\ 41.84 $\rightarrow$ 56.17
	\end{itemize}
\end{frame}

\begin{frame}
	\frametitle{成果概述}
	系统调用相关改进
	\begin{itemize}
		\item [\tri] 添加了时间相关模块和系统调用
		\item [\tri] 添加了进程间通信相关模块和系统调用
		\item [\tri] 完善了文件系统和 IO 相关系统调用
		\item [\tri] 修复了一些系统调用的 bug
		\item [\tri] 共提交 17 个 Pull Request,涉及 35 个系统调用
	\end{itemize}
\end{frame}

\begin{frame}
	\frametitle{成果概述}
	功能移植相关进展
	\begin{itemize}
		\item [\tri] shell 可以运行外部命令
		\item [\tri] 可以使用 sh 命令执行简单的 shell 脚本
		\item [\tri] 可以使用 GCC 编译简单的 C 程序
	\end{itemize}
\end{frame}

\section{写在后面}
\begin{frame}
	\frametitle{后续任务}
	\begin{itemize}
		\item [\tri] 完善 sys\_symlink 系统调用
		\item [\tri] 完善 sys\_epoll 系统调用
		\item [\tri] 完善 sys\_mmap 系统调用
		\item [\tri] 完善信号机制
		\item [\tri] 完善对 shell 和 GCC 的测试
		\item [\tri] 移植 Rust 工具链
		\item [\tri] 添加网络相关功能
	\end{itemize}
\end{frame}

\begin{frame}
	\frametitle{致谢}
	感谢陈渝老师和向勇老师为我们提供这样一个学习和交流的平台
	\\
	感谢王润基学长细心的指导、耐心的 Code Review
	\\
	感谢助教们细致的答疑解惑
	\\
	感谢队友们在合作过程中给予的帮助
\end{frame}

\begin{frame}
	\frametitle{尾声}
	祝愿我们的 OS Tutorial Summer of Code 越办越好
\end{frame}

\begin{frame}
	\frametitle{尾声}
	谢谢观看
\end{frame}

\end{document}
